% Thesis Preface ------------------------------------------------

\begin{preface}      %this creates the heading for the preface

  My time as a graduate student at Cambridge was beyond intellectually stimulating. In the laboratory, I worked with a variety of experimental techniques, such as magnetic resonance imaging, spectroscopy and psychophysics. I had the wonderful opportunity of combining experiments with modeling. But most importantly, I had the privilege of working on the fascinating topic of stereopsis. Some of the most remarkable minds of the last two millenia --- Euclid, Da Vinci, Kepler, Newton, Descartes, and more --- have been at some stage bewildered by binocular vision and stereopsis. They were mostly struggling with the geometry of binocular vision (i.e. how the light is captured by the left and right eyes and how might that relate to depth in the environment). This is now well known. What we don't know yet is how the brain uses the signals captured by the left and right eyes to estimate depth. This is the central point of the thesis.
  
  Given that ultra-high field magnetic resonance imaging was not yet available in Cambridge, data acquisition for the experiments reported in Chapter 4 and 5 was performed in collaboration with the University of Nottingham and the University of Maastricht, respectively. The remaining contents of the thesis result from my own work, guided by my advisors Dr. Andrew Welchman and Prof. Zoe Kourtzi. I have not submitted any parts of the thesis for any other degree in the University of Cambridge or any other institution. The thesis does not exceed the word limit established by the Degree Committee for the Faculty of Biology.
  
\end{preface}

% ------------------------------------------------------------------------

%%% Local Variables: 
%%% mode: latex
%%% TeX-master: "../thesis"
%%% End: 
